\textbf{Plan estratégico de Facultad}

La Facultad de Ciencias Físicas y Matemáticas desarrolló un plan estratégico a partir del año 2002, que incluyó un conjunto de acciones para mejorar su posicionamiento. Luego, a partir de un análisis prospectivo y de la comparación con instituciones líderes en el mundo, formuló un plan estratégico -FCFM 2030- que busca que en el año 2030 la Facultad se sitúe entre las 100 mejores escuelas de ingeniería en el mundo y entre las tres mejores en América Latina. Dicho plan estratégico establece una serie de directrices para mejorar el desempeño en diferentes ámbitos tales como docencia de pregrado, investigación, extensión y administración, a niveles de estándares internacionales. 

El principal objetivo del Proyecto FCFM 2030 es convertir a la Facultad en una institución de clase mundial, reconocida por su liderazgo en ciencia, tecnología e innovación, impulsando una investigación multidisciplinar de vanguardia para hacer frente a los desafíos y necesidades de la sociedad, y proporcionando una experiencia educativa de excelencia, con impacto y responsabilidad social, dentro del país y de Latinoamérica. Como objetivos específicos, se tiene:

\begin{enumerate}
\item Por medio de la participación activa de la comunidad y por una difusión constante de sus actividades, desarrollar las acciones y adaptar la estructura institucional de la FCFM para fomentar, supervisar y evaluar la innovación y el espíritu emprendedor en ciencia y tecnología.
\item Incentivar la investigación y educación multidisciplinaria, y así aumentar el impacto y la pertinencia de las actividades de investigación y educación, generando capacidades mejoradas de innovación.
\item Respecto a los planes de estudios: profundizar y mejorar la metodología CDIO, añadiendo dos componentes principales: (1) evaluación y (2) enseñanza de la innovación y el espíritu emprendedor, y mejorar la calidad de vida y la experiencia de los estudiantes a través de una nueva dirección de Asuntos Estudiantiles.
\item Respecto a investigación: mejorar el impacto de la investigación sobre la sociedad nacional y mundial mediante el fomento de actividades multidisciplinarias y el fortalecimiento por el número y el foco de los programas de postgrado.
\item Impulsar un profundo cambio cultural y organizacional para fomentar la innovación y el espíritu emprendedor en ciencia y tecnología (en todos los niveles) por medio de una oficina de Transferencia Tecnológica e Innovación, y un Laboratorio de Innovación y Emprendimiento Científico-Tecnológico.
\item Mejorar la internacionalización por medio de una oficina de Asuntos Exteriores.
\item Mejorar y acelerar los procesos de divulgación.
\end{enumerate}
